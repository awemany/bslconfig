\NeedsTeXFormat{LaTeX2e}

\documentclass[11pt,a4paper]{article}
\usepackage{microtype}
\usepackage{hyperref}

\begin{document}

\title{Configurable Maximum Block Size Limit}

\author{reddit.com/u/Peter\_\_R \and reddit.com/u/awemany}
\date{August 17th, 2015}
\maketitle

\section*{Introduction} 

A lot has been said about the Bitcoin maximum blocksize limit
 (BSL)\cite{bitcoin-scalability} in many heated arguments during the
 last few months. For solving this issue, a number of proposals have
 been made, such as BIP100\cite{bip100-draft}, BIP101\cite{bip101} and
 BIP102\cite{bip102}.

Common to all the BSL proposals is an algorithm (which in some cases is
as simple as a predetermined schedule) that will be used for
determining the blocksize.

The most prominent Bitcoin implementation is the \texttt{Bitcoin/Core}
client\cite{bitcoin-core}, which has been considered to be the
\emph{reference implementation} until recently. The developers
working on this client agreed on a process of reaching human and
political consensus before implementing a change to the Bitcoin
system, to avoid overriding user's wishes with controversial changes.

Unfortunately, none of the proposals dealing with blocksize have
reached consensus so far. Arguably, progress on an agreement regarding
BSL is deadlocked due to irreconcilable positions of many
\texttt{Bitcoin/Core} developers and the wider community.


\section*{The proposal} 
To solve this contention, we argue that the best approach forward is
 for all involved parties on the developer side to \emph{completely
 step back from actually making a concrete decision on the BSL, and
 leave it completely up to the users of the software to set this
 limit}.

To be more specific, we argue that the currently implemented 1MB limit
in \texttt{Bitcoin/Core}, which was introduced\cite{1mb-introduction} as a temporary anti-spam measure
by Bitcoin's original author will be replaced with a
freely selectable limit that every node operator can set individually.

We propose a command line argument for the \texttt{bitcoind}
application named \texttt{-blockmaxsizelimit} that takes a numerical
integer argument (measured in bytes) and which will replace what is
formerly the hard-coded \texttt{MAX\_BLOCK\_SIZE} constant in the
code.

For the QT GUI client \texttt{bitcoin-qt}, we propose an edit-box for
replacing the \texttt{MAX\_BLOCK\_SIZE} constant, appearing in a modal
dialog upon start up of the application, necessary to be confirmed upon
each start.

This edit-box respectively command line argument will be evaluated at
program start and the set BSL is deemed to be valid until the program
terminates.

We further propose that until January 11th 2016, specifying the\linebreak
\texttt{-blockmaxsizelimit} argument is optional and will default to
the current value of 1MB. Equally, the corresponding setting for
\texttt{bitcoin-qt} should be pre-filled with a value of $1000000$
until this date.

At the switchover date of January, 11th 2016 00:00 UTC (as determined
by the local system clock), the default setting of 1MB will
disappear, and the \texttt{-blockmaxsizelimit} argument will become
mandatory. In the QT GUI, the edit box will become empty and mandatory
to be filled for any further operation of the software.

\paragraph{Bitcoin packages.} We suggest that agreement to this
proposal includes urging any Bitcoin package maintainers to refrain
from providing a default. Setting the limit should be left to the
node operator or system administrator.

\paragraph{The 32MB network message limit.} Agreement to this proposal
should be taken as an endorsement for eventually making all parts of Bitcoin
sufficiently flexible to support \emph{any} integer blocksize
limit. Notably, no historical, technical limit in the code should be
assumed to be an in any way intended BSL.


\section*{Rationale} Much has been said about the dangers of fracturing
the Bitcoin network through introducing incompatible clients that are
unable to talk to each other and form a common consensus.

We share these concerns to some extend, observe however, that
participants in the Bitcoin ecosystem have a strong incentive to
create a working consensus network.

\paragraph{Decentralized decision making.} Bitcoin is commonly seen as
being all about \emph{decentralized consensus}. The reference
implementation creates a central place for \emph{information
exchange}, but we propose that it should refrain from becoming a
central place for \emph{decision making}. Especially so for
controversial decisions. 

\paragraph{User-centered software development.} We further argue that
software is developed for the needs of users and in any sane software
development environment, the software implementation follows, as much
as it is reasonably possible, the needs of the users and not the
personal preferences of developers.

In light of the widespread controversy, \texttt{Bitcoin/Core} developers can
avoid the power as well as taking the responsibility to make this
decision by putting it back to the decentralized set of users of the
network.

\paragraph{Harnessing collective wisdom.}  Further, due to the
controversy involving this variable, and our admission of the lack of
knowledge about its perfect value, we expect a node operator community
that is well-educated about this issue to be the best bet at
decentrally approximating the true, but unknown, optimal value.


\paragraph{Selecting a switchover date in the future.} Having a
switchover date in the future gives enough time to allow to educate
full node operators sufficiently on this important change. We chose
the above date because it is coincident with BIP101's
activation. BIP101 already achieved some popularity through the
\texttt{Bitcoin/XT} node software variant\cite{bitcoin-xt} and the principle of
least surprise suggests to pick the same date. It should be noted that
this proposal is in no way meant as an endorsal of BIP101.


\section{Further extensions} 
The above description of a static limit that is valid during the
runtime of \texttt{bitcoind} and \texttt{bitcoin-qt} could be replaced
with a limit that can be set live, for example through a secure
JSON-RPC call.

An external program, a `BSL governor' can then be used to specify the
maximum block size limit of the full node. Many such governors could
exist, completely independent of any particular Bitcoin protocol
implementation.

Such a scheme would also allow to implement all current BIPs
concerning the blocksize with an external program, further
decentralizing the decision on block size.

\section*{Attribution}
It should be made clear that what is
discussed here is in no way an original idea of the authors of this
proposal. It has been discussed earlier, for example on
the \texttt{Bitcoin/Core mailing} list\cite{idea-config}.


\begin{thebibliography}{9}
\bibitem{bitcoin-scalability}
  Bitcoin Scalability, Bitcoin Wiki, \url{https://en.bitcoin.it/w/index.php?title=Scalability&oldid=57432}

\bibitem{bip100-draft}
  BIP100 v0.8.1 draft, Jeff Garzik, \url{http://gtf.org/garzik/bitcoin/BIP100-blocksizechangeproposal.pdf}

\bibitem{bip101}
  BIP101, Gavin Andresen, \url{https://github.com/bitcoin/bips/blob/master/bip-0101.mediawiki}

\bibitem{bip102}
  BIP102, Jeff Garzik, \url{https://github.com/bitcoin/bitcoin/pull/6451}

\bibitem{bitcoin-core}
  The \texttt{Bitcoin/Core} client, \url{https://github.com/bitcoin/bitcoin}

\bibitem{bitcoin-xt}
 The \texttt{Bitcoin/XT} client, \url{https://bitcoinxt.software/}

\bibitem{idea-config}
  [bitcoin-dev] Bitcoin Core and hard forks, Cory Fields, \url{http://lists.linuxfoundation.org/pipermail/bitcoin-dev/2015-July/009544.html}

\bibitem{1mb-introduction}
  Introduction of the 1MB constant, Satoshi Nakamoto, \url{https://github.com/bitcoin/bitcoin/commit/a30b56ebe76ffff9f9cc8a6667186179413c6349}

\end{thebibliography}
\end{document}
